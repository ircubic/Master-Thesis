\documentclass[]{report}
\usepackage[round,sort&compress,authoryear]{natbib}
\usepackage[usenames,dvipsnames]{color}
\usepackage{listings}
\usepackage{hyperref}
\usepackage{hypcap}
\usepackage{graphicx}
\usepackage{amssymb,amsmath}


% Set up some colors
\definecolor{gray92}{gray}{.92}
\definecolor{gray75}{gray}{.75}
\definecolor{gray45}{gray}{.45}

% Set up some PDF options and reference coloration
\hypersetup{
    pdftitle={Selected Topics},
    pdfauthor={Daniel E. Bruce},     % author
    colorlinks=true,       % false: boxed links; true: colored links
    linkcolor=red,          % color of internal links
    citecolor=black,        % color of links to bibliography
    filecolor=magenta,      % color of file links
    urlcolor=cyan           % color of external links
}

% listings settings
\lstset{
  breaklines=true,
  framerule=0.5pt,
  linewidth=\textwidth,
  numbers=left,
  showstringspaces=false
}

\lstdefinestyle{console}
{
  numbers=none,
  basicstyle=\bf\ttfamily,
  backgroundcolor=\color{gray92},
  frame=lrtb,
}

\lstset{style=console}


% Use "normal" paragraph separation
\setlength{\parskip}{1.3ex plus 0.2ex minus 0.2ex}
\setlength{\parindent}{0pt}

% Nicer margin comments
\let\oldmarginpar\marginpar
\renewcommand\marginpar[1]{\-\oldmarginpar[\raggedleft\footnotesize #1]%
{\raggedright\footnotesize #1}}


%
% Document begins here
%
\begin{document}
\title{Selected Topics}
\author{Daniel E. Bruce}
\date{\today}
\maketitle

\tableofcontents

\chapter{Introduction}

This document covers a selection of the most relevant previous work related to
my Thesis.

There are two main areas that have to be covered, specifically the topics of
Artificial Intelligence in games, divided into AI implementation techniques and
AI entertainment value, and Automatic Programming.

There is also some work that relates to the specific topic of using Automatic
Programming to improve game AI, either for skill or entertainment value.

\chapter{Related work}
\label{cha:related-work}

\section{Game AI}
\label{cha:conclusion}

The work existing on the topic of Artificial Intelligence in games can be roughly
divided between work describing various implementation techniques, and work
describing how to make game AI more entertaining within the constraints of
existing implementation techniques.

The line between these two subtopics is very fuzzy and gradual, so some
judgment has been employed when putting the work into one category or the other,
in some cases the choice is purely symbolical, such as when the work describes a
new implementation technique created for the express purpose of increasing
entertainment value, such as in \citet{Mateas02}.

\subsection{Techniques}
\label{sec:techniques}

\subsection{Entertainment value}
\label{sec:entertainment-value}

\section{Automatic Programming}
\label{cha:autom-progr}

\chapter{Conclusion}
\label{cha:conclusion-1}

\bibliographystyle{abbrvnat}
\bibliography{topics}

\end{document}

% LocalWords:  ADATE
