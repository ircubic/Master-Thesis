\cleardoublepage
\chapter{Introduction}
\label{intro} The introduction to your document should lead your
readers into your paper and give them an idea of what to expect. It
should not be simply a restatement of the abstract even though it
will contain some of the same material.

Introductions often do the following:
\begin{itemize}
  \item State the subject of your document as clearly as possible

  \item Define the problem you are addressing, your approach to the problem, and why this problem is important
  \item State the purpose of your document
  \item Define the scope of your document
  \item Provide necessary and relevant background information
  \item Give an outline of the rest of the document
\end{itemize}


Because the introduction leads your reader into your document, try
to begin with a general statement about the topic before moving on
to specific issues. This strategy will help make the topic
accessible to your readers, especially those who are not specialists
in the field. Illustrations, like Figure~, often
help to introduce the reader to the problem.

More stuff in Chapter~\ref{rel}.

\section{ Lists } Lists of items can be enumerated and itemized.

\begin{enumerate}
\item This is the first item in an enumerated list.
\item   This is the second.  There is no limit to the length of items in a list.
    A single item can be several sentences long.  You can have lists nested
    within other lists.  You can put equations, tables, and figures in
    lists as well.
\item   This is the third item.
\end{enumerate}
That list was enumerated.  The following list is itemized.
\begin{itemize}
\item   Item number one in an itemized list.
\item   Itemized lists work just like enumerated list...
\item   Item $\cal C$.
\end{itemize}
%%% Local Variables:
%%% TeX-master: "main"
%%% End:
