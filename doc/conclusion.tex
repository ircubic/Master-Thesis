\cleardoublepage
\chapter{Conclusions}
% \label{dc} Wrap up (in separate sections, or integrated in one final
% section):

% \begin{description}
%   \item[Discussion] Explain in the discussion section of your document information
% presented in the results section, commenting on significant data and
% experience produced by the study.
%   \item[Conclusions] Include a conclusion as the final part of the body of your document.
%     Because some readers of documents, particularly managers,
%     will sometimes not read the entire document but, instead, focus on the conclusion,
%     this part of the document should summarize all essential information necessary for your audience's purpose.
%     In your conclusion:
%     \begin{itemize}
%       \item Relate your findings to the general problem and any specific objectives posed in your introduction.
%       \item Summarize clearly what the report does and does not demonstrate.

%       \item Include specific recommendations for action or for further research.
%       Sometimes these recommendations will constitute a separate section of a document.

%     \end{itemize}
%   \item[Recommendations] Include appropriate and specific recommendations as part of your conclusion or,
%   in feasibility and recommendation reports, as a separate section preceding the conclusion.

% Many types of scientific and technical documents conclude by
% pointing to further action. Research reports often recommend further
% studies to confirm tentative explanations or to answer questions
% presented in the discussion section. Feasibility and recommendation
% reports always have one or more specific recommendations as the
% principal aim of the document.

% Recommendations should always be specific and appropriate to the
% document's audience. Separate each specific recommendation. Often
% authors present recommendations in bulleted or numbered lists.
% Organize recommendations either in the order of importance or in the
% logical order of development.
% \end{description}
%%% Local Variables:
%%% TeX-master: "main"
%%% End:
