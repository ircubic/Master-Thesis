\cleardoublepage
\chapter{Design (You may use a more specific title in your report)}
\label{design}  Outline a scenario or two, derive some more detailed
use cases, then present the resulting design. Use illustrations
generously. The design is typically implemented, which is described
in Chapter~\ref{implement}.

Typesetting of mathematical and scientific documents were one of the
original reasons for developing \LaTeX, and as you see, it's working
quite well, ...however, the learning curve is a bit steep. Here are
some samples to start with.

\section{ Examples of mathematical expressions }

Never start a paragraph with an equation! Equation (\ref{eq:abc})
says $ \alpha = \beta \gamma \delta $.
\begin{equation}
  \alpha = \beta \gamma \delta
  \label{eq:abc}
\end{equation}
Equations are automatically numbered by \LaTeX.  You can refer to an
equation by its number if you label the equation. e.g., Equation
(\ref{eq:abc}). Labeling equations is optional.

The equation-formatting capabilities of \LaTeX\  are highly touted!
The following is an important equation in solid mechanics. It also
shows how to do sub-scripts, super-scripts, and fractions.
\begin{equation}
  I_{zz} = \int_{-b/2}^{b/2} \int_{-h/2}^{h/2} y^2 dy dx = \frac{b h^3}{12}.
  \label{eq:mom-inert}
\end{equation}

Other mathematical symbols are available, such as $\approx$, $\pm$,
$\times$, $\div$, $\propto$, $\leq$, $\geq$, $\ll$, $\gg$, $\neq$,
$\nabla$, $\Re$, $\Im$, $\flat$, $\sharp$, $\partial$, $\infty$,
$\sin$, $\log$, $\arctan$, $\heartsuit$, and many, many more.
Mathematical objects, like arrays, vectors, and matrices can be
created as well. See any text on \LaTeX\  for more details regarding
mathematical formulas\footnote{Making Greek letters is a piece of
$\pi$!}.




\section{Proof of the Area of a Circle Formula $A = \pi r^2$}
\newtheorem{prf}{Theorem}


\begin{prf}
The area of circle with radius $r$ is $\pi r^2$.
\end{prf}

\noindent {\bf Proof:} The equation of a circle centered at the
origin is

$$
x^2 + y^2 = r^2,
$$

\noindent where $r$ is the radius.  We  write $y$ in terms of the
variable $x$ and the constant $r$:

$$
\frac{x^2}{r^2} + \frac{y^2}{r^2} = 1
$$
$$
\frac{y}{r} = \sqrt{1-\frac{x^2}{r^2}}
$$
$$
y= r\sqrt {1-\frac{x^2}{r^2}}
$$

By symmetry, the area of a circle centered at the origin is four
times the area of the circle between $(0,0)$ and $(r, 0)$ above the
$x$-axis.  We can integrate to find the area ($A$):

$$
A = 4r\int_0^r \sqrt {1-\frac{x^2}{r^2}}\, dx
$$

To evaluate the antiderivative of $\displaystyle\sqrt
{1-\frac{x^2}{r^2}}$, we make the substitutions:

$$
x = r \sin \theta
$$
$$
\theta = \arcsin \frac{x}{r}
$$
$$
dx = r\cos \theta\, d\theta
$$

Thus, our integral becomes:

$$
A=4r\int_0^r \sqrt {1-\frac{x^2}{r^2}}\, dx = 4r\int_0^{\pi/2}
r\sqrt{1-\sin^2 \theta} \cos \theta\, d\theta
$$

 We can use the trigonometric identity $1 - \sin^2 \theta = \cos^2 \theta$:

$$
A=4r\int_0^{\pi/2} r\sqrt{1-\sin^2 \theta} \cos \theta\, d\theta=
4r^2\int_0^{\pi/2} \cos^2 \theta\, d\theta
$$

We then apply $\cos^2 \theta = \frac{1}{2}(1 + \cos 2\theta)$:

\begin{eqnarray*}
4r^2\int_0^{\pi/2} \cos^2 \theta\, d\theta &=& 4r^2\int_0^{\pi/2}  \frac{1}{2}(1 + \cos 2\theta) \,d\theta\\
& = & {2r^2\theta}\Bigg{|}_0^{\pi/2} + 2r^2\int_0^{\pi/2} \cos 2\theta \,d\theta\\
                                  & = & \pi r^2 + 2r^2(\sin2\theta)\Bigg{|}_0^{\pi/2}\\
                                  & = & \pi r^2
\end{eqnarray*}

Thus, the area of a circle with radius $r$ is $\pi
r^2$.\hfill$\blacksquare$
%%% Local Variables:
%%% TeX-master: "main"
%%% End:
